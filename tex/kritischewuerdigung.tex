\section{Kritische Würdigung}

\subsubsection*{Versuchsdurchführung}

\todo{Diese subsubsection verbessern, ausformulieren}

Für die dritte Mode haben wir offensichtlich leider den falschen Messbereich ausgewählt.
Für höhere Temperaturen wandert die Resonanzkurve weiter Richtung Null, sodass sie eher im Messbereich noch dargestellt wird.
Man erkennt dies außerdem gut im Plot mit den über die Temperatur aufgetragenenen Resonanzfrequenzen:
Was die Fehler der Messpunkte in y-Richtung (also der Fehler auf $\omega_R$) angeht, haben alle Punkte relativ kleine Fehler - die Fehlerbalken sind nahezu nicht sichtbar.
Für die dritte Mode erkennt man für die erste Hälfte der Messwerte (im kälteren Temperaturbereich also) jedoch deutlich größere Fehler.
Hier war der Fit einfach nicht sehr erfolgreich aufgrund der unklaren Messpunkte.
Je wärmer, desto besser werden die Fehler.

\subsubsection*{Einordnung der Fiteigenschaften}

\todo{Diese subsubsection verbessern, ausformulieren}

Beim Erstellen der Fits an die aufgenommenen Daten geht es uns darum, das in der Einleitung beschriebene theoretische Modell möglichst gut an die aufgenommenen Daten anzupassen.
Daraus können wir zum einen schließen, ob das Modell die Realität tatsächlich beschreibt - und zum anderen erhalten wir so auch weitere Informationen über das System:
Wir können etwa die Resonanzfrequenzen der Schwingung genau feststellen.

Letzteres funktioniert sehr gut.
Die Fehler auf $\omega_0$ sind winzig.

Für $\omega_{0}$ passt die Lorentzkurve sehr gut an die Daten.
Dies ist in den Plots ersichtlich und wird auch durch den guten $\chi_{red}^2$-Wert belegt.
Dennoch liegt die Fitwahrscheinlichkeit nur bei unter 10\%.

\subsubsection*{Vergleich mit den theoretischen Verhältnissen}

In der Einleitung Gleichung \ref{eq:VerhaeltnisseEigenfrequenzen} wurden theoretische Verhältnisse der Eigenfrequenzen $\frac{\nu_1}{\nu_0}$ und $\frac{\nu_2}{\nu_0}$ eingeführt.
Diese Werte lassen sich nun mit den von uns gewonnenen experimentellen Ergebnissen vergleichen.
Die Ergebnisse sind in Tabelle \ref{tab:VergleichEigenfrequenzen} aufgetragen.

\minipage{\linewidth}
    \begin{center}
        \captionsetup{type=table}
        \begin{adjustbox}{max width=\linewidth, keepaspectratio}
            \begin{tabular}{llllll}
            \toprule
            $\left ( \frac{\nu_1}{\nu_0} \right )_{theo}$ & $\left ( \frac{\nu_1}{\nu_0} \right )_{exp}$ & Abweichung & $\left ( \frac{\nu_2}{\nu_0} \right )_{theo}$ & $\left ( \frac{\nu_2}{\nu_0} \right )_{exp}$ & Abweichung \\
            \midrule
            6.267 & 6.22590 $\pm$ 0.00016 & 0.041 $\overset{\wedge}{=}$ 260 $\sigma$ & 17.548 & 17.620 $\pm$ 0.098 & 0.072 $\overset{\wedge}{=}$ 0.73 $\sigma$ \\
            \bottomrule
            \end{tabular}
        \end{adjustbox}
        \captionof{table}{Vergleich von experimentellen und theoretischen Verhältnissen zwischen den Eigenfrequenzen}
        \label{tab:VergleichEigenfrequenzen}
    \end{center}
\endminipage

Wir erkennen für $\frac{\nu_1}{\nu_0}$ eine große Abweichung von 260 $\sigma$.
Diese ist allerdings damit zu begründen, dass die Fehler $\omega_0$ der einzelnen Moden so gering sind.
Die relative Abweichung mit $\frac{0.041}{6.267} = 0.7\%$ ist nicht besonders groß und so kann davon ausgegangen werden, dass die Eigenfrequenzen tatsächlich gefunden wurden.

\subsubsection*{Interpretation der Abschätzung systematischer Fehler}

\todo{Diese subsubsection verbessern, ausformulieren}

Wir betrachten die residuals in den Abbildungen der Amplituden.
Es wird deutlich, dass zu keiner der Resonanzfrequenzen systematische Abweichungen erheblich sichtbar wären.

\subsubsection*{Vergleich der berechneten Güte für die erste Mode}

Wir haben die Daten der Amplitude $A$ verwendet und einerseits mit den aus dem Fit gewonnenen Parametern die Güte $Q_{Fit} = 59.06 \pm 0.19$ errechnet.
Andererseits haben wir die full width (FW) auf der Höhe des $\frac{1}{\sqrt{2}}$-fachen Maximums der Lorentzkurve zu $FW = \SI{34}{\second^{-1}}$ bestimmt und anschließend die Güte
\begin{align}
    Q_{FW} = \frac{\omega_R}{FW} = 59.05
\end{align}

berechnet. Da nur ein qualitativer Vergleich angestellt werden soll, verzichten wir an dieser Stelle auf die Abschätzung eines Fehlers für $Q_{FW}$.

Die oben genannten Werte bedeuten eine Abweichung von $Q_{FW}$ zu $Q_{Fit}$ um \SI{0.01} absolut - beziehungsweise \SI{0.05}{\sigma}.
Wir können also davon ausgehen, dass die Bestimmung der Güte mit beiden Methoden gleich gut funktioniert - je nachdem, welche Methode sich gerade besser anbietet.

\subsubsection*{Temperaturabhängigkeit der Resonanzfrequenzen}

\todo{Temperaturabhängigkeit der Resonanzfrequenzen erklären - Stichwort Schallgeschwindigkeit}

\todo{Temperaturabhängigkeit unterscheidet sich für die verschiedenen Moden}

\subsubsection*{Analyse möglicher Fehlerquellen}

\todo{Text zu möglichen Rauschursachen laut Versuchsanleitung?}

\subsubsection*{PyVISA als Alternative zu LabVIEW}

\todo{Text fertig schreiben}

\subsubsection*{Simulation des kompletten Versuchs}

\todo{Text fertig schreiben}
