\section{Kritische Würdigung}
%
\subsubsection*{Verstärkerschaltung mit einem Transistor}
%
Die Dimensionierungen der Schaltungen im Vorfeld des Versuchs konnte ohne Probleme durchgeführt werden.
Es wurden keine Fehler festgestellt.
\par
Die beiden kennengelernten, grundlegenden Transistor-Schaltungen Emitterschaltung und Kollektorschaltung haben sich im Experiment als erfolgreich bewiesen.
Wie in der Vergleichstabelle zu sehen sind die Abweichungen zwischen den experimentell bestimmten Werten und den Theoriewerten stets kleiner als $3 \sigma$.
Meist sogar deutlich kleiner.
Damit ist keine signifikante Abweichung festzustellen.
\par
Bei der Untersuchung des Phasenunterschieds konnte die vorhandene Invertierung (beziehungsweise deren Nichtvorhandensein) der Spannung durch die Schaltung anhand der Aufnahmen mit dem Oszilloskop festgestellt werden.
Die geplotteten Frequenzgänge stimmen mit den erwarteten Graphen, wie in der Versuchsanleitung zu sehen, überein.
\par
Sowohl die Bestimmung der Kabellänge als auch die Bestimmung der Wellenimpedanz eines Kabels haben sehr gut funktioniert.
Es sind keine signifikanten Abweichungen von den Theoriewerten festzustellen.
\par
Die vorhandenen Messinstrumente und Geräte während der Versuchsdurchführung waren verständlich bezüglich ihrer Verwendung und haben ohne technische Probleme funktioniert.
Die Qualität der Kabel ist jedoch als mangelhaft einzustufen und birgt unnötige zusätzliche Arbeit beim Identifizieren der Messprobleme.
Hier könnten für zukünftige Durchführungen des Versuchs mit geringem Aufwand und geringer Investition bessere Versuchsbedingungen geschaffen werden.
%
\subsubsection*{Aufbau eines PID-Reglers zur Drehzahlregelung eines Motors}
%
Der Aufbau der Schaltung stellte sich umfangreicher aber nicht unbedingt komplizierter dar.
Alle qualitativen Ergebnisse unserer Messungen decken sich mit den im Vorfeld gegebenen Erklärungen zu Reglern in der Versuchsanleitung.
\par
Nachdem wir die optimalen Parameter für die PID-Regelung gefunden haben, ließen sich keine Probleme bei der Motorsteuerung feststellen.
Besonders beeindruckend ist das Folgen des Istwerts bei manuell eingestelltem internem Sollwert.
Sobald wir am Potentiometer den Widerstand geändert haben, folgt der Istwert automatisch und flüssig.
%
\subsubsection*{Fazit und zusammenfassender Eindruck}
%
Insgesamt ist der Versuch damit als äußerst positiv einzustufen.
Die theoretische Vorbereitung war thematisch interessant und nicht zu umfangreich oder zu anspruchsvoll - obwohl das Thema an sich bisher im Studium eher nicht behandelt wurde.
Es könnte vorher klarer gemacht werden, in welchem Umfang Formeln oder Schaltbilder abgefragt werden und/oder auswendig gelernt werden sollen.
Die Durchführung konnte ohne auftretende Probleme stattfinden.
Die von uns hergestellten Platinen dürfen mit nach Hause genommen werden und dort entweder als Erinnerung aufgehoben oder für andere Projekte recycled werden.
In der Auswertung stellt sich heraus, dass sich unsere theoretischen Erwartungen experimentell bestätigt haben.
\par
Wir haben damit in diesem Versuch praktisch und theoretisch neue Kenntnisse im Umgang mit elektronischen Bauteilen gewonnen, die sich in Zukunft als nützlich erweisen können.
%
