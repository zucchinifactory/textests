\section{Durchführung}

\subsubsection*{Einstellungen an den Geräten}

Wir überprüfen zunächst die Einstellungen am Lock-In Verstärker.
Dabei nutzen wir hauptsächlich die vorgegebenen Werte aus der Versuchsanleitung \cite{Anleitung} sowie einzelne Modifikationen unter der Anleitung des Assistenten.
Die endgültigen Einstellungen sind in Tabelle \ref{tab:EinstellungenVerstaerker} aufgetragen.

\minipage{\linewidth}
    \begin{center}
        \captionsetup{type=table}
        \begin{adjustbox}{max width=\linewidth, keepaspectratio}
            \begin{tabular}{lllll}
            \toprule
            Vorderseite    & ~        & ~                       & Rückseite      & ~                \\
            \midrule
            SIGNAL FILTERS & BANDPASS & IN                      & SINE AMPLITUDE & \SI{1}{\volt}    \\
            ~              & LINE     & IN                      & VCO RANGE      & \SI{100}{\hertz} \\
            ~              & LINEx2   & IN                      & ~              & ~                \\
            SENSITIVITY    & ~        & \SI{100}{\micro\volt}   & ~              & ~                \\
            DYN RES        & ~        & LOW                     & ~              & ~                \\
            TIME CONSTANT  & PRE      & \SI{300}{\milli\second} & ~              & ~                \\
                           & POST     & \SI{0.1}{\second}       & ~              & ~                \\
            REFERENCE      & PHASE    & -90 DEG                 & ~              & ~                \\
            ~              & MODE     & 2f                      & ~              & ~                \\
            ~              & TRIG     & $\sim$                  & ~              & ~                \\
            CHANNEL 1/2    & OFFSET   & OFF                     & ~              & ~                \\
            ~              & EXPAND   & X1                      & ~              & ~                \\
            \bottomrule
            \end{tabular}
        \end{adjustbox}
        \captionof{table}{Einstellungen am Lock-In Verstärker}
        \label{tab:EinstellungenVerstaerker}
    \end{center}
\endminipage

\subsubsection*{Entwicklung des LabVIEW-Skripts}

Der Tutor gibt zu Beginn des Versuchs eine kurze, hilfreiche Einweisung zu LabVIEW.
Zudem stehen uns mehrere LabVIEW-Beispielskripte auf der Versuchs-Homepage zur Verfügung.

Wir beginnen mit der Erstellung eines Arrays von Werten im Bereich beginnend bei \SI{0}{\volt} bis zur maximalen Heizspannung des Peltier-Elements.
Die darauf folgende Schleife über die Elemente dieses Arrays beginnt mit Einfügen einer Kommentarzeile zum Logging der eingestellten Heizspannung.
Anschließend wird die Heizspannung mit Befehl \texttt{X5} gesetzt.
Aus unseren Erfahrungen mit dem Aufheizverhalten des Peltier-Elements können wir mit großer Sicherheit ableiten, dass sich nach \SI{900000}{\milli\second}, also \SI{15}{\minute}, die neue Temperatur eingestellt hat.

Es folgt die Erstellung eines weiteren Arrays.
Dieses Array beinhaltet die einzustellenden Frequenzen für den Frequenzgenerator.
Erneut beginnen wir eine Schleife und loggen das Einstellen der Frequenz.
Der verwendete Befehl zum Setzen der Frequenz ist \texttt{FREQ}.

Wir beginnen nun den Abschnitt der Aufnahme von Messdaten, welcher für die oben beschriebenen Einstellungen jeweils 20 mal wiederholt wird.
Mit \texttt{X1}, \texttt{QX} und \texttt{QY} fragen wir die Signaldaten an und schreiben diese in die Messdatei.
Um eine Beeinflussung der Messungen untereinander auszuschließen, ist vorher eine Wartezeit von \SI{500}{\milli\second} eingebaut.

Nach Abschluss aller Messungen wird die Heizspannung des Peltier-Elements auf \SI{0}{\volt} zurückgesetzt.
So kann der Versuchsaufbau wieder auf Raumtemperatur abkühlen.

Das endgültige LabVIEW-Skript ist in Abbildung \ref{fig:LabVIEWSkript} dargestellt.

\subsubsection*{Heizspannung des Peltier-Elements}

Parallel zur Entwicklung unseres LabVIEW-Skripts für die eigentliche Messung, untersuchen wir die Grenzen der Heizspannung des Peltier-Elements.
Dabei erhöhen wir die Spannung immer wieder um \SI{0.1}{\volt} und warten anschließend bis sich eine neue Gleichgewichts-Temperatur eingependelt hat.
Die Temperatur bestimmen wir in regelmäßigen Abständen manuell per Gleichung 3.1 der Versuchsanleitung \cite{Anleitung}.
Als Grenze sollen \SI{60}{\celsius} nicht überschritten werden.

Wir können so für die spätere komplette Messung den Bereich der Heizspannung des Peltier-Elements auf \SIrange{0}{0.9}{\volt} bestimmen.

Nachdem die Bestimmung des Bereichs der Heizspannung abgeschlossen ist, wird die Betriebsspannung des Peltier-Elements wieder auf \SI{0}{\volt} gesetzt.

\subsubsection*{Resonanzfrequenz der ersten Mode als Anhaltspunkt}

Wir wollen in einer ersten Messung die Resonanzfrequenz der ersten Mode bestimmen.
Diese sollte im Vergleich zu den höheren Moden den Vorteil haben, dass sie eine größere Amplitude besitzt.
Wir erhoffen uns davon, dass sie besser auffindbar ist.

Grundsätzlich tragen sowohl eine große Anzahl an Messpunkten, als auch ein großer Messbereich zu einer langen Messdauer bei.
Um die Dauer der Messung in einem akzeptablen Rahmen zu halten, ist es daher sinnvoll die vorher gemachten Abschätzungen aus der Einleitung zu verwenden.
Wir wissen so, in welchem Bereich die Resonanzfrequenz der ersten Mode etwa zu erwarten ist.
Siehe entsprechenden Abschnitt \ref{eq:AbschaetzungEigenfrequenzen}.

Leider lassen sich unsere zunächst aufgenommenen Daten einer detaillierteren Messung rund um den erwarteten Bereich für die Resonanzfrequenz der ersten Mode nicht verwenden.
Es stellt sich heraus, dass statt der erwarteten Resonanzkurve eine Oberschwingung der \SI{50}{\hertz} Netzfrequenz des Stromnetzes vermessen wurde.
Der Messbereich muss also für eine erfolreiche Suche nach der Resonanzfrequenz der ersten Mode noch deutlich erweitert werden.

\subsubsection*{Abschließende Messung}

Mit den aktualisierten Suchbereich lässt sich ein weiterer Peak auffinden.
Er gehört zu einer Kurve von Messpunkten mit dem charakteristischem Verlauf einer Resonanzkurve.
Mit diesem ersten Wert für die Resonanzfrequenz der ersten Mode lassen sich mithilfe von Gleichung \ref{eq:VerhaeltnisseEigenfrequenzen} die Frequenzen für höhere Moden bestimmen.
Wir definieren schließlich die drei Messbereiche \SIrange{143}{173}{\hertz}, \SIrange{975}{1005}{\hertz} und \SIrange{2757}{2787}{\hertz} für die Messung der Resonanzkurven der ersten drei Moden und starten die abschließende Messung, welche die endgültigen Messdaten für unsere Auswertung liefert.
