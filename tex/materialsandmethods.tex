\section*{Materials and methods}
%
\subsection*{Setup of detectors}
%
Our setup uses two scintillation counters on either side of the radioactive sample, the left one labeled \enquote{1}, the right one labeled \enquote{2}. All is enclosed in a detection chamber made from lead bricks.
%
We set the voltage for counter 1 to $\SI{540 \pm 2}{\volt}$ and for counter 2 to $\SI{490 \pm 2}{\volt}$.
Using the $^{60}\text{Co}$ sample (labeled \enquote{NN258}) we familiarize ourselves with the detector signals and examine the functions of the used components using the oscilloscope.
%
\par
%
Then we route signal 1 through the amplifier (\enquote{Coarse Gain} set to $16$, \enquote{Fine Gain} set to maximum) into the \textbf{Timing Single Channel Analyzer} (TSCA) with a fully open energy window.
The output then is used to trigger the \textbf{Linear Gate} while the original signal is also routed through the \textbf{Delay Amplifier} (set to \SI{2.75}{\micro\second}) into the input of the gate.
On the output we can now take the spectrum.
%
\subsection*{Measurement of simple spectra}
%
We connect the Linear Gate output to the \textbf{Multi Channel Analyzer} (MCA), which delivers its signals to the LabVIEW software (threshold of $200$).
%
\par
%
Thereupon, we measure the spectra of different sources for approximately \SIrange{300}{400}{\second} each.
Sample $^{60}\text{Co}$ (\enquote{NN258}) shows two clear peaks and an continuum with multiple spikes as seen in figure \ref{fig:Spectrum60Co}.
Sample $^{137}\text{Cs}$ (\enquote{NN256}) shows one peak and Compton continuum as seen in figure \ref{fig:Spectrum137Cs}.
Sample $^{22}\text{Na}$ (\enquote{NN261}) shows two high energy peaks and one very intense peak caused by electron-positron annihilation as seen in figure \ref{fig:Spectrum22Na}.
%
\par
%
We also looked at the spectra of $^{54}\text{Mn}$ (\enquote{NN260}, \enquote{AC9339}) and $^{133}\text{Ba}$ (\enquote{NN255}) and take a night measurement with the sample holder removed and the detection chamber closed.
%
\subsection*{Coincidence measurement setup}
%
To test coincidence the second scintillation detector is set up the same way as the first.
Its output signal is used to trigger the Linear Gate.
This way we will only register the signal of counter 1 if counter 2 is being triggered at the same time as well.
%
\par
%
We insert the $^{137}\text{Cs}$ sample and record the spectrum shown in figure \ref{fig:137Cskoinz1}, which has three distinct peaks.
From left to right they are the backscatter peak, Compton edge and photo peak.
%
\par
%
To measure the randomly occurring coincidence, the Delay Amplifier is removed from signal 1 and  the delay on the TSCA is set to its maximum value.
The two coincident lower energy peaks lose relative height compared to the high energy photo peak.
%
\subsection*{Improved coincidence measurement with TPHC}
%
To further improve the circuit, we use a \textbf{Time to Pulse Height Converter} (TPHC) which translates the time difference between two input signals to a pulse with respective height.
It is controlled by the two signals through their respective TSCAs.
The delay of the TSCA for signal 2 is set to $7.32$ (scale precision of \SI{0.1}{\micro\second}).
We observe one prominent peak caused the real coincidence.
Additionally, we vary the delay on the Delay Amplifier with values \SI{20}{\nano\second}, \SI{40}{\nano\second} and \SI{60}{\nano\second}.
%
\par
%
Now we use the time window settings of the TPHC to send a signal only if the time difference corresponds to a backscattered photon of the Compton scattering at the opposite scintillator.
In turn, this signal is used to trigger the Linear Gate.
Slowly adjusting the lower value and window size yields the best results when ULD is set to $1.0$ and LLD is set to $1.1$.
%
\subsection*{Coincidence measurement with time and energy window}
%
Finally we use the TSCA 2 to only allow signals resulting from backscattered photons to trigger the TPHC.
To achieve this, we setup scintillator 2 as in the initial setup and adjust the energy window of the TSCA so as to only show the backscatter peak.
The lower threshold is set to $0.5$ scale units, the window is set to $0.4$.
%
\par
%
With these settings we return to the previous layout and measure the spectrum shown in \ref{fig:comptonpeak}.
The only peak left belongs to the Compton edge which corresponds to the middle one of the three peaks in the former spectrum obtained without time and energy window.
%
\subsection*{Coincidence Measurement of $^{60}\text{Co}$ cascade decay}
%
Using the above described method, we want to show that the photons in the $^{60}\text{Co}$ cascade decay are emitted coincidentally.
We also note that the radioactive activity of the provided $^{60}\text{Co}$ sample has been determined on 01.08.2005 to $\SI{404}{\kilo\becquerel \pm 3 \percent}$.
%
\par
%
We set the energy window on both TSCAs to only trigger around the two photon peaks.
This is the case for a \enquote{Lower Level} of $3.2$ and a \enquote{Window} of $1.0$ for signal 1 and a \enquote{Lower Level} of $2.8$ and a \enquote{Window} of $1.0$ for signal 2.
Additionally, we note the time and total event number to calculate the photon rates.
For detector 1 we get \SI{87576}{events} in \SI{300}{\second}.
For detector 2 we get \SI{92712}{events} in \SI{317}{\second}.
%
\par
%
As before we set up the TPHC to take the time spectrum.
On TSCA2, the delay is set to $4.3$.
We obtain the spectrum with \SI{3174}{events} in \SI{361}{\second}.
For calibration we again use different delays in the stop signal of the TPHC.
%
\par
%
To examine the coincidence we narrow the energy window on TCSA2 to only show the peak with higher energy (\enquote{Lower Level} of $3.15$ and \enquote{Window} of $0.45$).
Now only the lower energy peak is visible proving the coincident relationship.
We measure \SI{4796}{events} in \SI{1422}{\second}.
%
