\section{Introduction}
%
\subsection{Radioactive decay}
% Atomkerne mehrere Stadien außer stabil.
Besides the stable state, there are various conditions an atomic nucleus can exist with.
% Falls instabil, radioaktive Strahlung.
If its state is unstable, the nucleus decays and emits radiation in the process.
% Mehrere Möglichkeiten für nach Zerfall: für uns wichtig beta und Übergänge zwischen den Anregungsstadien.
This is called radioactive decay and among others it may result in multiple modes of $\beta$ decay as well as transitions between different excitation states of the same nucleus.
%
\subsection{Modes of $\beta$ decay}
% beta-
The process of a decaying neutron resulting in the emission of an electron and an electron antineutrino is called \textbf{$\beta^{-}$ decay}.
A newly formed proton remains with the nucleus.
% beta+
The related process would be \textbf{$\beta^{+}$ decay} during which the nucleus obtains a neutron and emits a positron and an electron neutrino following a decaying proton.
% Electron Capture
Another mode is \textbf{electron capture} (EC) which takes place if an orbiting electron is being captured by a proton of the nucleus. Consequently, an electron neutrino is being emitted and the nucleus remains in an excited state.
%
\subsection{Transitions between excitation states}
% Isomeric transition
A $\gamma$ ray is emitted if an orbital electron of an excited nucleus wanders to a lower energy state.
This is called \textbf{isomeric transition} (IT).
% Internal conversion
The process of \textbf{internal conversion} (IC) means the transfer of energy of an excited nucleus onto one of its shell's electrons which results in the ejection of said electron from the atom.
%
\subsection{Interaction of $\gamma$ rays with matter}
% Falls gamma auf Materie trifft gibt es Wechselwirkung. Kann ausgenutzt werden für Spektrometer.
If $\gamma$ rays encounter matter, there are various possibilities of interaction. These interactions can be detected and used to build spectrometers.
%
\subsection{Photo effect}
%
\subsection{Compton scattering}
%
\subsection{Pair production}
%
\subsection{Scintillators}
%
\subsection{Scintillation spectrometer}
%
% Wir nutzen anorganischen Szintillator.
Even though organic scintillators exist as well, anorganic scintillators are better suited to detect $\gamma$ rays due to their higher atomic number and density. [TODO: REFERENZ PRAKTIKUMSANLEITUNG]
% Funktionsweise Szintillator
