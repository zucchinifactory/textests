\section{Introduction}
% Versuch geht um radioaktiven Zerfall und die Untersuchung der entstehenden Gammaquanten mithilfe Szintillationsspektroskopie
This article is about our experiment on radioactive decay of different radioactive elements and the study on the subsequently appearing $\gamma$ rays with the aid of scintillation spectroscopy.
%
\subsection{Radioactive decay}
% Atomkerne mehrere Stadien außer stabil.
Besides the stable state, there are various conditions an atomic nucleus can exist with.
% Falls instabil, radioaktive Strahlung.
If its state is unstable, the nucleus decays and emits radiation in the process.
% Mehrere Möglichkeiten für nach Zerfall: für uns wichtig beta und Übergänge zwischen den Anregungsstadien.
This is called radioactive decay and among others it may result in multiple modes of $\beta$ decay as well as transitions between different excitation states of the same nucleus.
%
\subsection{Modes of $\beta$ decay}
% beta-
The process of a decaying neutron resulting in the emission of an electron and an electron antineutrino is called \textbf{$\beta^{-}$ decay}.
A newly formed proton remains with the nucleus.
% beta+
The related process would be \textbf{$\beta^{+}$ decay} during which the nucleus obtains a neutron and emits a positron and an electron neutrino following a decaying proton.
% Electron Capture
Another mode is \textbf{electron capture} (EC) which takes place if an orbiting electron is being captured by a proton of the nucleus. Consequently, an electron neutrino is being emitted and the nucleus remains in an excited state.
%
\subsection{Transitions between excitation states}
% Isomeric transition
A $\gamma$ ray is emitted if an orbital electron of an excited nucleus wanders to a lower energy state.
This is called \textbf{isomeric transition} (IT).
% Internal conversion
The process of \textbf{internal conversion} (IC) denotes the transfer of energy of an excited nucleus onto one of its shell's electrons which results in the ejection of said electron from the atom.
%
\subsection{Interaction of $\gamma$ rays with matter}
% Falls gamma auf Materie trifft gibt es Wechselwirkung. Kann ausgenutzt werden für Spektrometer.
If $\gamma$ rays encounter matter, there are various possibilities of interaction. These interactions can be detected and used to build spectrometers.
%
\subsection{Photo effect}
% Photoeffekt tritt auf wenn Photonen passender Energie auf Materie treffen und somit Elektronen rausschlagen
The photo effect occurs when photons of adequate energy strike matter which results in emission of electrons.
% Gamma überträgt seine gesamte Energie auf ein Elektron und schlägt es damit aus der Hülle eines Atoms heraus
A $\gamma$ quantum transfers its complete energy onto one electron and thereby liberates it from the atom.
%
\subsection{Compton scattering}
% Stoß von gamma mit Elektron heißt Compton
Inelastic scattering of a $\gamma$ quantum with an electron is called Compton scattering.
% Photon verliert Energie
The photon transfers energy onto the electron which consequently increases the photon's wavelength.
% Energie hängt vom Winkel ab
The amount of energy which is transferred onto the electron depends on the scattering angle $\theta$:
\begin{align}
    \label{eq:PhotonNachStoss}
    \begin{split}
        {E_{\gamma}}'(\theta) &= E_{\gamma} - \frac{E_{\gamma}}{1 + \left ( 1 - \cos(\theta) \right ) \frac{E_{\gamma}}{m_{\text{e}} c^2}}
    \end{split}
    \\
    \label{eq:ElektronNachStoss}
    \begin{split}
        E_{\text{e}}(\theta) &= E_{\gamma} - {E_{\gamma}}'(\theta)
    \end{split}
\end{align}
where $E_{\gamma}$ and ${E_{\gamma}}'$ are the energies of the photon before and after the collision, $E_{\text{e}}$ is the energy of the electron after the collision, $m_{\text{e}}$ is the mass of the electron and $c$ is the speed of light. \cite{AnleitungZusatz2}
%
\subsection{Pair production}
% Falls Energie von gamma größer als Ruheenergie dann Paarbildung möglich was das Photon zu Elektron Positron Paar macht
If the energy of the $\gamma$ quantum exceeds \SI{1022}{\kilo\electronvolt}, which is the rest mass of an electron-positron pair, pair production may take place and the photon is converted into an electron-positron pair.
The involved nucleus ensures conservation of momentum.
\cite{AnleitungZusatz1}
%
\subsection{Scintillators}
% Angeregter Szintillator emittiert sichtbares oder UV-Licht
When a scintillator is being excited with high energy photons, it emits light of visible or ultraviolet range.
% Wir nutzen anorganischen Szintillator weil besser laut Aussagen in Anleitung
Even though organic scintillators exist as well, anorganic scintillators are better suited to detect $\gamma$ rays due to their higher atomic number and density. \cite{Anleitung}
%
\subsection{Scintillation spectrometer}
% Auftreffende gamma geben Energie an Elektronen
Impacting $\gamma$ rays transfer their energy onto electrons of the scintillator material as described above.
% Diese Elektronen gehen ins Leitungsband oder bilden Exziton
In turn, these electrons leave the valence band for the conduction band while creating an electron hole as well or - in case of an insufficient amount of transferred energy - form an exciton.
% Danach diffundieren die Energieträger zu den angelegten Fehlstellen
Either way, electrons, electron holes and excitons diffuse through the semiconductor until they encounter one of the inhomogeneities which have actively been created by doping.
% Dort existieren Energielevel zwischen Valenz- und Leitungsband sodass die Energieträger dort Energie abgeben können, was Photonen erzeugt
They act as an emission center by creating energy levels between the levels of the conduction and valence band - therefore promoting the energy carriers to dispose their energy which results in generated photons.
% Die Lichtblitze werden mit Photomultiplier detektiert
These flashes of light can then be detected with a photomultiplier tube.
