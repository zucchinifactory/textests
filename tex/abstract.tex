\section*{Abstract}
%
This article is about the experiment on coincidence spectroscopy as part of the \enquote{Fortgeschrittenenpraktikum}.
It is meant to give a practical introduction into the methods of scintillation detectors.
The theoretical processes of radioactive decay and the interaction of the resulting $\gamma$ rays with matter are tested experimentally.
The coincidence method is used to detect the correlation of photons emitted in decays of certain radioactive samples.
%
\par
%
We were able to quantitatively verify the underlaying theoretical physical processes.
Additionally, the validity of common physical models concerning the repective decay schemes could be confirmed.
Via combining two methods of energy and time selection, random coincidence could successfully be decreased in great significance.
