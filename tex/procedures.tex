\section{Procedures}
%
\subsection{Setup and initial measurements}
%
In the following text, the counter on the left (labeled "1") of our setup will be called "counter 1" and the one on the right (labeled "2") "counter 2".
%
\par
%
We set the voltage for counter 1 to $\SI{540 \pm 2}{\volt}$ and for counter 2 to $\SI{490 \pm 2}{\volt}$.
The cables which carry the signals from the spectroscopes are plugged directly into the oscilloscope.
We insert the $^{60}\text{Co}$ sample (labeled "NN258") into the detector chamber and close the sliding door.
On the oscilloscope screen, pulses of different height and length can be seen.
Exemplary screenshots for a quantitative analysis are listed in the appendix:
Figure \ref{fig:OsciSignal1} shows a signal from detector 1 with a pulse height of $\SI{50 \pm 5}{\milli\volt}$ and a pulse length of $\SI{100 \pm 10}{\micro\second}$.
Detector 2 is shown in figure \ref{fig:OsciSignal2} with two signals with a pulse height of $\SI{240 \pm 20}{\milli\volt}$ and a pulse length of $\SI{50 \pm 5}{\micro\second}$ or a pulse height of $\SI{50 \pm 5}{\milli\volt}$ and a pulse length of $\SI{30 \pm 5}{\micro\second}$, respectively.
The uncertainties for the above measurements stem from reading the scale only.
%
\par
%
For the next step, we connect the signals of both counters to the corresponding amplifiers according to the circuit schematic in figure \ref{fig:Schaltung1}).
Modulator "Coarse Gain" is set to $16$.
Modulator "Fine Gain" is set to its maximal value and won't get changed in the future to minimize reproducibility errors.
Both amplified signals are connected to the oscilloscope and produce the output as shown in figure \ref{fig:OsciSignal1and2}.
Furthermore, the amplifier is set to "Negative Input" and "Unipolar" (because the input signal is already bipolar) which results in the bipolar signal as shown in figure \ref{fig:BipolarSignal}.
%
\par
%
Now, signal 1 is being used to control the \textbf{Timing Single Channel Analyzer} (TSCA) as seen in the schematic in figure \ref{fig:Schaltung2}.
We connect the amplified signal 1 to input 1 and the TSCA positive output to input 2 of the oscilloscope.
The TSCA window is set to fully open.
This results in a square wave as seen in figure \ref{fig:ExampleTSCA}.
The TSCA sends a rectangular pulse of constant height when the input signal crosses the zero line.
With the minimal energy knob we can control which input pulse height is needed to make the TSCA send a square wave pulse.
The energy window knob then provides an upper threshold for the height of the signals to trigger an output.
%
\par
%
Using the TSCA output to control the \textbf{Linear Gate} as depicted in the circuit schematic in figure \ref{fig:Schaltung3} results in the output seen in figure \ref{fig:TSCAplusLG}.
The delay of the \textbf{Delay Amplifier} is set to \SI{2.75}{\micro\second}, which ensures that the gate is being opened exactly when the original amplified signal reaches the gate.
%
\subsection{Measurements with further samples}
%
To analyze further samples we connect the Linear Gate output to the \textbf{Multi Channel Analyzer} (MCA), which delivers its signals to the LabVIEW software.
Within the application's interface, the detection threshold is set to $200$.
%
\par
%
Thereupon, we measure the spectra of five different sources for approximately \SIrange{300}{400}{\second} each.
Sample $^{60}\text{Co}$ ("NN258") shows two clear peaks and an continuum with multiple spikes as seen in figure \ref{fig:Spectrum60Co}.
Sample $^{137}\text{Cs}$ ("NN256") shows one peak and Compton continuum as seen in figure \ref{fig:Spectrum137Cs}.
Sample $^{54}\text{Mn}$ ("NN260", "AC9339") shows one strong peak and Compton continuum, more jagged as seen in figure \ref{fig:Spectrum54Mn}.
Sample $^{133}\text{Ba}$ ("NN255") shows three more broad and less separated peaks as seen in figure \ref{fig:Spectrum133Ba}.
Sample $^{22}\text{Na}$ ("NN261") shows two high energy peaks and one very intense peak caused by electron-positron annihilation as seen in figure \ref{fig:Spectrum22Na}.
%
\par
%
During the night measurement, the sample holder is removed from the detector chamber and the sliding door is closed.
For \SI{68898}{\second}, the counters and the software are left running to detect background radiation.
The resulting spectrum can be seen in figure \ref{fig:SpectrumNight}.
%
\par
%
The positions of the distinct peaks and edges as well as their full width half maxima (FWHM) are determined and added to the respective spectrum figure.
%
\subsection{Coincidence measurement setup}
%
The second scintillation detector is set up the same way as the first with the amplifier's "Coarse Gain" at $16$ and "Fine Gain" at maximum value as well as the TSCA's window fully open.
Its output signal is used to trigger the Linear Gate as seen in circuit schematic in figure \ref{fig:Schaltung4}.
This way we will only register the signal of counter 1 if counter 2 is being triggered at the same time as well.
%
\par
%
We insert the $^{137}\text{Cs}$ sample and record the spectrum shown in figure \ref{fig:137Cskoinz1}, which has three distinct peaks.
From left to right they are the backscatter peak, Compton edge and photo peak.
The spectrum of counter 2 is slightly shifted to the right (around 100 channels), because the scintillators have different inherent reinforcement.
%
\subsection{Random coincidence}
%
To measure the randomly occuring coincidence, the Delay Amplifier is removed from signal 1.
The delay on the TSCA is set to its maximum value.
The resulting signal is shown in figure \ref{fig:137Cskoinz2}.
The two lower energy peaks lose relative height compared to the high energy photo peak.
%
\subsection{TPHC setup}
%
To further improve the circuit, we use a \textbf{Time to Pulse Height Converter} (TPHC) which translates the time difference between two input signals to a pulse with respective height.
The circuit is set up as seen in the schematic in figure \ref{fig:Schaltung5}.
The delay of the TSCA for signal 2 is set to $7.32$ on the potentiometer knob which has a scale precision of \SI{0.1}{\micro\second}.
Additionally, we vary the delay on the Delay Amplifier with values \SI{20}{\nano\second}, \SI{40}{\nano\second} and \SI{60}{\nano\second}).
With these additional measurements, we are able to calibrate the resulting time spectrum.
The resulting spectra are shown in figures \ref{fig:137CsTPHC}, \ref{fig:137CsTPHC20ns}, \ref{fig:137CsTPHC40ns} and \ref{fig:137CsTPHC60ns}.
%
\subsection{Improved coincidence measurement with TPHC}
%
We use the time window settings of the TPHC to send a signal only if the time difference corresponds to a backscattered photon of the Compton scattering at the opposite scintillator.
In turn, this signal is used to trigger the Linear Gate as seen in the schematic in figure \ref{fig:Schaltung6}.
Slowly adjusting the lower value and window size yields the best results when ULD is set to $1.0$ and LLD is set to $1.1$.
We obtain the time spectrum which is depicted in figure \ref{fig:137CsTPHC_geschnitten_neu}.
The corresponding energy spectrum is shown in figure \ref{fig:137CsmitTPHC_gated}.
%
\par
%
During the experiment, there were technical dificulties with the equipment and the circuit had to be set up fresh from the start again.
Therefore, our spectrum for this specific measurement has slightly different settings for the time window and delay time.
However, the observable results were qualitatively the same.
We just did not record the former spectrum with our new settings for a second time.
%
\subsection{Coincidence measurement with time and energy window}
%
Now we want to use the TSCA 2 to only allow signals resulting from backscattered photons to trigger the TPHC.
To achieve this, we return to the schematic according to figure \ref{fig:Schaltung3} for scintillator 2 and adjust the energy window so as to only show the backscatter peak.
The lower threshold is set to $0.5$ scale units, the window is set to $0.4$.
The resulting spectrum is shown in figure \ref{fig:signal2eneriewindow}.
%
\par
%
After changing the circuit back to the schematic shown in figure \ref{fig:Schaltung6} we measure the spectrum shown in \ref{fig:comptonpeak}.
The only peak left belongs to the Compton edge which corresponds to the middle one of the three peaks in the former spectrum obtained without time and energy window.
%
\subsection{Coincidence Measurement of $^{60}\text{Co}$ cascade decay}
%
Using the above described method, we want to show that the photons in the $^{60}\text{Co}$ cascade decay are emitted coincidentally.
The radioactive activity of the provided $^{60}\text{Co}$ sample has been determined on 01.08.2005 to $\SI{404}{\kilo\becquerel \pm 3 \percent}$.
%
\par
%
We set the energy window on both TSCAs to only trigger around the two photon peaks.
This is the case for a "Lower Level" of $3.2$ and a "Window" of $1.0$ for signal 1 and a "Lower Level" of $2.8$ and a "Window" of $1.0$ for signal 2.
Additionally, we note the time and total event number to calculate the photon rates.
For detector 1 we get \SI{87576}{events} in \SI{300}{\second} and the spectrum as shown in figure \ref{fig:60CoEnergiewindow1}.
For detector 2 we get \SI{92712}{events} in \SI{317}{\second} and the spectrum as shown in figure \ref{fig:60CoEnergiewindow2}.
%
\par
%
We return to the circuit schematic shown in figure \ref{fig:Schaltung5} to obtain a new time measurement.
On TSCA2, the delay is set to $4.3$ with a scale precision of \SI{0.1}{\micro\second}.
We obtain the spectrum shown in figure \ref{fig:60CoZeitspektrum} with \SI{3174}{events} in \SI{361}{\second}.
For energy calibration we again use different delays in the stop signal of the TPHC.
The corresponding spectra are shown in figures \ref{fig:60CoZeitspektrum20ns}, \ref{fig:60CoZeitspektrum40ns} and \ref{fig:60CoZeitspektrum60ns}.
%
\par
%
To examine the coincidence qualitatively, we narrow the energy window on TCSA2 to only show the peak with higher energy.
This means a "Lower Level" of $3.15$ and a "Window" of $0.45$.
The adjusted spectrum is shown in figure \ref{fig:60CoEnergiewindow2-peak2}.
%
\par
%
Finally, we return to the circuit schematic shown in figure \ref{fig:Schaltung6}.
The only peak left in the spectrum is the coincident lower energy peak.
The spectrum is shown in figure \ref{fig:60CoKaskadenEnergiewindowBeiPeak2} with \SI{4796}{events} in \SI{1422}{\second}.
%
